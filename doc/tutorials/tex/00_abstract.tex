SBMLsimulator is a fast, accurate, and intuitive program for dynamic simulation and heuristic parameter optimization of models encoded in the \SBML.
To ensure the high reliability of this software, its internal simulation library had been
benchmarked against the entire \SBMLTestSuite and all models from the \BioModels database.
SBMLsimulator includes the extensive collection of nature-inspired heuristic optimization procedures for efficient model calibration from the framework \EvA.
SBMLsimulator provides an easily usable \GUI and several command-line options to be suitable for large-scale batch processing and model calibration.
SBMLsimulator runs on all platforms that provide a standard \JVM for a desktop environment.
SBMLsimulator is an open-source program, which is based on the open-source libraries \JSBML and \SBSCL, both of which can be obtained separately and used via their \API.

Version~2.0 of SBMLsimulator comes with a variety of additional features for the visualization of biological networks in the form of \SBGN Process Diagrams (PD) powered by the yFiles library for graph drawing.
Upon loading \SBML files, SBMLsimulator either draws the layout of the biological network as given in the file or (if the \SBML file does not contain any graph layout information), it automatically generates \SBGN-style diagrams.
In either case, the network can be accessed in the graph view.
Simulation results, as well as loaded experimental measurement data from \CSV files, can be mapped to the networks to display the data in their full biological context.
SBMLsimulator can generate animations for time-series data that can then be exported to various movie file formats or as snapshot images for selected time points.

For a more intuitive description of the features described in this users' guide, a series of video tutorials are available at \href{https://youtube.com/c/systemsbiology/}{\faYoutube/c/systemsbiology/}.
For many chapters in this tutorial, corresponding video tutorials are available online.
You can find the direct links to those videos in the text.

This document is based on an earlier version of a users' guide to SBMLsimulator by \citet*{Doerr2014a}.