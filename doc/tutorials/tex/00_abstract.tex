The program SBMLsimulator has been developed in the context of the Virtual
Liver project to simulate models of metabolic, gene regulatory, and signal
transduction processes in individual cells or ensembles of cells. It thus can
be used at the compartmental, cellular or multicellular level. The program
reads biological models encoded in SBML files (Systems Biology Markup Language)
and efficiently calculates the evolution of all model components, such as
(cellular) species or compartment sizes, influences of external parameters, or
events. To this end, SBMLsimulator interprets the model as an ordinary
differential equation system. The user can choose among nine different solvers
for the numerical integration of the system.

In many cases, biological models contain quantities with uncertain values.
These values must be estimated by calibrating the model with respect to given
experimental data. To facilitate this complicated procedure, SBMLsimulator
comes with the optimization toolbox EvA2 that provides a large collection of
nature inspired heuristic optimization procedures, e.g., evolution strategies,
genetic algorithms, differential evolution, or particle swarm optimization.

A further important aspect of the program is its simple and self-explanatory
graphical user interface. For large-scale computation, all functions can also
be used in a command-line mode. SMBLsimulator is completely written in Java�
and uses the JSMBL package for its internal data structures. It has been tested
on Windows 7, Linux (Ubuntu 10.04 LTS), and MacOS 10.6.8 Snow Leopard with JDK
1.6. SBMLsimulator can be downloaded and used as a stand-alone program, or
launched as a Java� Web Start program. The computational core of the program
can be obtained separately.

In summary, SBMLsimulator has been developed as a convenient program for
efficient numerical simulation and large-scale model calibration of biological
models.
